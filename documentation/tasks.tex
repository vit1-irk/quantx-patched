%!TEX encoding = cp1251
\documentclass[12pt]{article}
\usepackage[cp1251]{inputenc}
\usepackage[russian]{babel}
\usepackage{literat}
\usepackage[a4paper,inner=3cm,outer=3cm,top=3cm,bottom=3cm]{geometry}
\usepackage[dvipsnames]{xcolor}
\usepackage{euler}
\usepackage{sectsty}
\usepackage{hyperref}
\hypersetup{
  colorlinks=true,
  linkcolor=OliveGreen,
  pdfstartview=FitH
}
\renewcommand{\Re}{\mathop{\mathrm{Re}}\nolimits}
\renewcommand{\Im}{\mathop{\mathrm{Im}}\nolimits}
\allsectionsfont{\raggedright}

\begin{document}
%\hafamily

\section*{Задачи для решения с помощью программы Квант\thanks}
\footnotetext{\parbox[t]{.9\columnwidth}
{Этот документ является частью пакета Квант
(\href{http://quantx.sourceforge.net}
      {http://quantx.sourceforge.net}).

\smallskip
\copyright{} Новосибирский государственный университет, 1984--1992

\copyright{} Ткаченко О.А., Ткаченко В.А., Коткин Г.Л., 2011

\smallskip
Каждый имеет право воспроизводить, распространять и/или вносить изменения в
настоящий документ в соответствии с условиями
\href{http://www.gnu.org/licenses/fdl.html}
     {GNU Free Documentation License}.
}}

\noindent
{\large О.А.Ткаченко, В.А.Ткаченко, Г.Л.Коткин}

\tableofcontents

\newpage
\hypertarget{characterization}{}\section[О заданиях]{О заданиях}
\hypertarget{purpose}{}\subsection{Цель практикума}
Эти задачи, решаемые в режиме %интерактивного
вычислительного эксперимента
с готовой программой, придуманы, чтобы сделать квантовую механику <<ручной>>
и наглядной, прояснить
ее понятия и закономерности, развить Вашу
интуицию и образное мышление, так необходимые физикам.\footnote[1]
{Мы начинаем с простой одномерной (1D) квантовой механики,
которая важна с образовательной
точки зрения и стала
интересной для экспериментаторов в связи с появлением реальной
инженерии наномасштабных 1D-потенциалов. Квантовые явления
в соответствующих объектах разнообразны и иногда
используются с большой выгодой.
Например, они работают в ультра-компакт\-ных
полупроводниковых  элементах сложных устройств, которые
находятся дома и в кармане у каждого. Забавно, что на
компьютере Вы изучаете 1D квантовую механику при
помощи электронных процессов, в которых она играет свою роль.}
В первой версии это было сделано в 1987\,г.\footnote
	{Коткин Г.Л., Ткаченко О.А., Ткаченко В.А.
	\emph{Лабораторные работы по квантовой механике. Вып.\ 1.
	Алгоритмы и задачи. Описание лабораторных работ.}
	Новосибирск, НГУ, 1987 (35 стр.).}
Еще мы хотим дать
%Вам
на будущее пример эффективной компьютерной технологии в сфере
образования и исследований.

\hypertarget{purpose1}{}\subsection{Тип задач}
Вполне ясным и строгим общим алгоритмом мы
за малое время моделируем гораздо больше квантовых объектов и
эффектов, чем при традиционном обучении. Все потенциалы, к
которым применяется этот алгоритм, являются достаточно
произвольными кусочно-постоянными функциями одной координаты. Некоторые простые задачи в каждой теме, например, про прямоугольную яму, помогают лучше понять
суть алгоритма и анализировать ответы, которые вы можете
получить аналитически. Однако в большинстве случаев
трудно или невозможно вывести конечные формулы, и
остаются лишь численные, хотя и точные (до ошибок округления),
решения. Плавные потенциалы с неограниченной точностью имитируются ступенчатыми при достаточно большом числе отрезков.
Экспериментируя с параметрами, получая разные графики и
числа необходимо найти и увидеть общие эффекты
для некоторых типичных условий. Детальное сопоставление потенциалов, энергетических спектров, волновых функций, коэффициентов прохождения и других величин позволяет провести в задачах четкую классификацию родственных, но все же разных явлений.\footnote[3]{
При рассмотрении стационарного рассеяния мы выделяем три типа резонансов по энергии --- надбарьерные резонансы, виртуальные уровни и квазиуровни. Первые два типа являются следствием резонансной интерференции при классически разрешенном движении, но относятся к физически противоположным потенциалам (барьеру, яме). Важно что надбарьерные резонансы при небольшой деформации потенциала сохраняются как таковые, тогда как нижние виртуальные уровни с $E>0$  могут преобразоваться в истинные уровни ($E_n<0$), либо наоборот. Ширина нижних надбарьерных резонансов как правило меньше ширины нижних виртуальных уровней.  В свою очередь, узкие квазиуровни являются результатом классически запрещенного движения ---резонансного туннелирования,
когда $|\Psi(x)|^2$ в резонаторе (включая крайние туннельные барьеры) очень напоминают их для уровней $E_n$, возникающих при удалении внешних границ этих барьеров на $\pm\infty$.
Есть также родственные каждому из этих трех типов квазистационарные состояния, но они имеют комплексные энергии $E_n+iG_n$, $G_n<0$, отвечают отсутствию приходящих волн, описывают распад--экспоненциальное уменьшение  $|\Psi(x)|^2$ со временем, и уходящие волны имеют комплексные волновые числа, т.е. экспоненциально нарастают по амплитуде при $x\to\pm\infty$. Как правило, $E_n$ из $E_n+iG_n$ лежат несколько ниже положения по $E$ аналогичных стационарных резонансных состояний. При классически разрешенном движении в плавном потенциале может вовсе не быть резонансов в стационарном рассеянии, но есть квазистационарные состояния!}

Центральным является вопрос, какие объекты мы при этом моделируем. На качественном уровне и для учебных целей ими могут быть свободная волна--частица, абстрактные <<атом>>, линейная молекула, кристалл. Если говорить о более серьезном описании реальности, то резкие разрывы потенциала могут быть сделаны лишь для частиц с большой длиной волны по сравнению с атомными размерами (субэлектронвольтных электронов и ультра-холодных нейтронов).\footnote[4]{Общий, хоть и не единственный, рецепт получения 1D-кусочно-постоянного потенциала для электронов такой.
%(его изобрели Ж.И.Ал\-феров и др. в Физико-техническом институте им. А.Ф.Иоффе).
Современные полупроводниковые технологии позволяют выращивать в едином кристалле чередующиеся слои разных составов и толщин с весьма совершенными <<гетерограницами>>. Управляя составами можно получить разные потенциалы, которые чувствует электрон в слоях, т.е. резкие разрывы потенциала на гетерограницах. Перпендикулярно к слоям может быть приложено внешнее однородное электрическое поле, что даст линейный ход потенциала $U(x)$ в слоях. Вводя в структуру заряд с концентрацией $\rho(x)$ можно еще больше усложнить потенциал.%Некоторый заряд почти всегда есть на верхней и нижней границах %выращенного многослойного кристалла.
Разумеется, при таком способе получения 1D-потенциала движение электрона вдоль слоев остается свободным. Более интересные эффекты относятся к движению поперек слоев. Еще проще делаются и аналогично работают твердотельные многослойники для ультра-холодных нейтронов.}


\hypertarget{character}{}\subsection{Характер работы}
Упор в задачах делается на самостоятельный поиск интересных
результатов без навязывания хода решения.
Сведений по управлению программой Квант в задачах нет ---
они вынесены в отдельную инструкцию, основы которой
можно быстро освоить, и обращаться к ней по мере необходимости.

В задачу полезно вчитываться, чтобы не пропустить главного.
Вопросы носят <<наводящий>> характер,
и иногда их больше, чем необходимо. % можно осилить за одно занятие.
Мы ожидаем от Вас самостоятельного выбора и учитываем
различие индивидуального опыта и склонностей.

Как минимум, предполагается, что Вы сможете описать
свои находки (словами и на
схематических графиках) товарищам и преподавателям.
Вопросы, живое общение в ходе и после занятий приветствуются,
как и совместный поиск правильных объяснений найденного.
Оценок не ставится ---
достаточно, если Вы откроете для себя интересное и новое.

\hypertarget{U0}{}\section{Свободное движение}
\hypertarget{TasksPsi_k}{}\subsection[Стационарные состояния]{Стационарные $\Psi$-функции}
Начнем с самого простого --- нормировки состояний непрерывного спектра. Задайте $U(x)=0$ (занулением глубины ямы) и скажите, чему равна площадь под кривой $|\Psi (x)|^2$ при $E>0$ и какова в нашем случае амплитуда падающей
волны?  Как наша нормировка связана с другими (например, на поток частиц)?

Сравнивая $Re\Psi(x)$, $Im\Psi(x)$ скажите, какой знак у нас имеет импульс частицы (откуда падает волна). Чем в данном случае является кривая 3D:$\Psi(x)$? Как зависит период на этих кривых от $E$?

\hypertarget{WP}{}\subsection{Волновой пакет}
Сформируйте пакет из
равноотстоящих по $E>0$ состояний:
$\Psi(x,t)=\Sigma \beta_j\cdot\Psi_j
(x)e^{-iE_jt}$, $E_{highest}\ge E_j\ge E_{lowest}$ (вес гармоник $\beta_j$ автоматически задан гауссовской
функцией по $E_j$).

Рассмотрите $|\Psi|^2$, $Re\Psi$,  3D:$\Psi$ сначала для случая одной  гармоники, потом двух близких по $E$ и, наконец, многих волн. Первый и второй варианты нужны, чтобы напомнить о временной зависимости стационарных состояний и увидеть различие фазовой и групповой скорости (какое оно из теории?). К чему ведет это различие в пакете из многих гармоник?

Проверьте что $\Psi(x,t)$ является периодической функцией $t$, т.е. содержит бесконечную серию <<пакетов>>.  (Из скоростей
вращения всех векторов $exp(-iE_jt)$ на комплексной плоскости
найдите наименьшее время, через которое они снова совпадут.) По числу гармоник и средней энергии предскажите и на числах проверьте расстояние между центрами <<пакетов>>. Как зависит ширина <<пакета>> от $E_{highest}-E_{lowest}$?


\hypertarget{step}{}\section{Потенциальная ступень}
\hypertarget{step_stationary}{}\subsection{Стационарное рассеяние частицы ступенью}

Все потенциалы, которые мы здесь изучаем, составлены из отрезков $U(x)=\textrm{const}$, в которых на частицу силы не действуют. Выразите силы через $|\Psi|^2$ в точках разрыва потенциала.

Рассмотрите столкновение частицы с энергией $E>0$ с простейшим препятствием --- бесконечно тонким ускоряющим промежутком, т.е. резкой ступенью вниз: $U(x<0)<0$. Почему при $x<0$ плотность вероятности является низкой горизонтальной линией, а справа имеются крупные осцилляции, в которых $|\Psi|^2_{\min}=|\Psi(x=0)|^2$? Почему осцилляции растут и растягиваются при приближении $E$ к нулю? Что при этом происходит с коэффициентом прохождения? Объясните форму кривых $Re\Psi(x)$, 3D:$\Psi(x)$.

Пусть теперь частица атакует %замедляющий бесконечно тонкий промежуток--
ступень с $U(x<0)>0$. При каких $x$, $E$ имеется стоячая волна, и в каких особенностях графиков видны ее проявления? То же самое для бегущей и затухающей волны.
Почему при $E>U_{\max}$ для осцилляций при $x>0$ выполняется условие $|\Psi|^2_{\max}=|\Psi(x=0)|^2$, т.е. плотность вероятности в ступеньке является высокой? При
какой $E$ она максимальна,
чему она равна и каков в данном случае коэффициент прохождения?
Сравните вид $Re\Psi(x)$, 3D:$\Psi(x)$ с тем, что было
при $U(x<0)>0$.

Найдите в $T(E)$ и координатных распределениях сходство с тем,
что дала бы классическая механика в подобной задаче, а также
принципиальные отклонения от классики. Например, докажите, что
давление на стенку получается одинаковым при $E<U_{\max}$. А что будет при $E>U_{\max}$?

Почему в квантовой
механике ступень (спуск) отражают частицы при $E>U_{\max}$? Укажите оптическую аналогию этому эффекту.



\hypertarget{WP_step}{}\subsection{Столкновение волнового пакета
с резкой ступенью}
Рассмотрите тот же самый пакет по энергии, как в задаче про \hyperlink{WP}{свободное движение}. Что и почему  происходит с
кривыми $|\Psi (x,t)|^2$, $Re \Psi (x,t)$, 3D:$\Psi(x,t)$, когда пакет подходит близко к ступени и далее со
временем?

Полезно смотреть движение пакетов из низких и больших энергий по сравнению с $U(x>0)$, а также  сопоставить со стационарным решением случай, когда пакет собран из состояний близких по энергии к $U_{\max}>0$.


%Что будет, если вместо ступени взять спуск, и пакетом охватить
%область энергий $U(x<0)<E<E_{\max}>0$ (падение с двух сторон).

\hypertarget{smooth_step}{}\subsection{Размытая ступень}
По аналогии с просветляющими покрытиями в оптике промоделируйте несколькими ступеньками плавный переход от
$U_{\max}$ до 0. При большом числе ступенек это может быть моделью потенциала в плоском
конденсаторе.

Как изменятся  $|\Psi (x)|^2$ в случаях
$E<U_{\max}$, $E>U_{\max}$, а также $T(E)$ по сравнению \hyperlink{step_stationary} {с резким
переходом}?  Каким нужно  взять размер области изменения  $U(x)$,
чтобы $T(E)$  стало почти таким же,  как для классических частиц?
Почему малые отличия $T$ от 1.0 хорошо заметны по осцилляциям
плотности вероятности в области, откуда падает частица?


\hypertarget{delta-well}{}\section[Имитация дельта-функционной ямы/барьера]{Имитация $\delta$-функционной ямы/барьера}
\hypertarget{delta-well/E<0}{}\subsection{Уровень в мелкой яме}
Управляя параметрами $a$, $U$ --- шириной и глубиной прямоугольной
потенциальной ямы, получите почти прижатый к потолку уровень
$E_0$.

Насколько
точно, если судить по числам $U$, $a$, $E_0$ и внешнему виду
$\Psi_0 (x)$, данная яма соответствует модели $\delta$-ямы?
(Аналитически из правил сшивки $\Psi (x)$ для $\delta$-ямы
найдите связь ее мощности c параметрами $U$, $a$, приняв во
внимание  поведение $\Psi_0 (x)$ внутри имитирующей ямы).

Соответствуют ли данной имитации построенные зависимости $E_0(z)$, где $z$ отвечает мощности ямы? На числах проверьте соотношение
неопределенностей Гейзенберга, и по поведению $\Psi_0 (x)$
предскажите внешний вид $|\phi_n(k)|^2$ в пределе $E_0 \to 0$.


\hypertarget{delta-well/E>0}{}\subsection{Коэффициент прохождения
в случае мелкой ямы}
Является ли мелкая широкая яма хорошей моделью  $\delta$-функци\-
он\-ной  ямы, если судить по дискретному спектру?  Тот же
вопрос для непрерывного спектра и $T(E)$.

Сравните коэффициенты
прохождения и координатные распределения для «$\delta$-ям»  и
«$\delta$-барьеров» одинаковой мощности.

\hypertarget{delta-well in step}{}\subsection[Дельта-яма в
потенциальной ступеньке]{<<$\delta$-яма>> в
потенциальной ступеньке}
Предскажите и численно найдите момент исчезновения уровня в
<<$\delta$-яме>> при повышении/опускании потенциала слева от ямы.
Какую форму имеют в этот момент $\Psi (x)$, $|\phi_n(k)|^2$?

Почему численно найденная $U_l=U(x<0)$ в момент исчезновения
уровня заметно отличается от предсказанного. Какой вид имеет
зависимость $E_0(U_l)$. Что происходит с $T(E)$ при изменении
$U_l<0$, $U_l>0$, пока в яме имеется уровень?

\hypertarget{delta-well}{}\section{Широкие яма/барьер}
\hypertarget{well}{}\subsection{Прямоугольная яма с несколькими
уровнями}
По форме $\Psi_n (x)$ при $E_n\rightarrow 0$ сообразите, при
каком условии на глубину и ширину ямы появляется очередной
уровень $n$, и как для него выглядит $|\phi_n(k)|^2$ в сравнении
с более глубокими уровнями $E_m$ ($m\le n$)?

Чем объясняется
почти одинаковый <<размах>> по вертикали для $\Psi_m (x)$ с разными
$m$ и малый <<размах>> для уровня прижатого к потолку ямы?  Как
предвидеть число, положение и ширину больших пиков импульсного
распределения, исходя из графиков $\Psi_m (x)$? Как зависят от
$m$ средняя кинетическая, средняя потенциальная энергия и сила, с
которой частица давила бы на стенки ямы. Чему равны эти величины
в пределе $E_n\rightarrow 0$?

Если увеличивать $U(x<0)$, то какие уровни это почувствуют
раньше и почему? Что и почему происходит с волновой функцией,
когда $U(x<0)$, наоборот, приближается сверху к соответствующему
уровню.

Какова плотность вероятности на левой стенке
ямы, если она известна для правой?
Проверьте на числах Ваше
предсказание для каких-нибудь уровней. Как найти силу, с
которой частица давит на левую стенку, если $U(x<0)=\infty$.
Каким в этом случае будет условие на появление в яме
единственного или очередного уровня?

Пусть на краю широкой ямы имеется узкое углубление, так чтобы
$E_0$ был ниже дна широкой части. Объясните вид $\Psi_n (x)$, $|
\phi_n(k)|^2$. Выборочно убедитесь, что сумма всех сил (с учетом
знака) в точках разрыва потенциала в любом состоянии есть 0.

\hypertarget{well-En-of-z}{}\subsection{Ход уровней при
расширении прямоугольной ямы}
Какова зависимость $E_n (z)$ (z-ширина ямы) пока уровни остаются
мелкими или становятся глубокими. Чему равна $dE_n/dz$ в момент
появления уровней? Рассматривая поведение $|\Psi_n (x)|^2$
опишите, как от $z$ зависит сила, с которой частица давит на
стенки ямы, сопоставьте это поведение с $E_n (z)$ и другим
выражением для силы ($-dE_n/dz$).

Чем для глубоких уровней
объяснить параллельность линий $E_n(z)$, когда $z$-глубина ямы?
Предскажите и проверьте, как меняются импульсные распределения $|
\phi_n(k)|^2$ с ростом ширины (глубины) ямы.

\hypertarget{well_T(z)}{}\subsection{Виртуальные уровни}
Что за пики на кривой $T(E)$, как и почему меняется их положение
и ширина с изменением ширины ямы $a$? Как в резонансах полного  прохождения связаны $\overline{|\Psi|^2}$ и $|\Psi|^2_{\max}$ внутри и вне ямы? Как вели бы себя аналогичные величины для стационарного потока классических частиц с энергией выше потолка ямы.

Как найти резонансные значения $E-U$, либо $a$ из рисунка $|\Psi(x)|^2$ в задаче про одну резкую ступень, и каков знак у этой ступени (спуск или подъем)? Почему эти резонансные значения совпадают с положениями уровней (выше некоторого) в яме с бесконечными стенками?

Что происходит с $T(E)$ возле
точки $E=0$ при расширении прямоугольной ямы перед появлением
нового уровня? Как это выглядит на зависимости $T(z)$ при малом
$E>0$, когда увеличивается $z$- ширина (глубина) ямы? Как и
почему меняется положение и ширина этих резонансов с изменением $E$? В чем сходство и различие координатных
распределений для резонансов $T$ по $z$ и для появившихся
уровней?

\hypertarget{WP_virtual}{}\subsection{Волновой пакет,
настроенный на виртуальный уровень}
Исходя из зависимости $T(E)$ сформируйте пакет в окне энергий, отвечающих ширине резонанса с виртуальным уровнем и оцените по его ширине характерное время $\tau$. Что и почему  происходит с
кривой $|\Psi (x,t)|^2$. Будет ли $|\Psi|_{\max}^2$ в яме больше, чем в пакете, который приближается к яме или уходит от нее?

Сравните $\tau$ с разностью времен прохождения пакета через экран с ямой и без нее. Какой знак и величину имел бы аналогичный эффект для классической частицы? Как на время  прохождения пакета влияет резонансная интерференция?
%Почему пакет пересекает яму быстрее, чем без нее? Как  интерференция внутри ямы замедляет прохождение по сравнению с  классической частицей?

\hypertarget{barrier}{}\subsection{Надбарьерные резонансы}
Для достаточно широкого потенциального барьера посмотреть
и объяснить $T(E)$, $T(z)$ с одновременной визуализацией $|\Psi
(x)|^2$ (особенно в точках максимума и минимума коэффициента
прохождения).

Как для надбарьерных резонансов (состояний полного прохождения) связаны $\overline{|\Psi|^2}$ и $|\Psi|^2_{\min}$ внутри и вне барьера? Как вели бы себя аналогичные величины для стационарного потока классических частиц с надбарьерной энергией? Как по графику $|\Psi(x)|^2$ из задачи про резкую ступень с $U(x<0)<0$ предсказать  значения $E-U$, $a$ для надбарьерных резонансов? Почему эти значения совпадают с положениями всех уровней (начиная с нижнего) в яме с бесконечными стенками?

\hypertarget{WP_overbarrier}{}\subsection{Волновой пакет и
надбарьерный резонанс}
Сформируйте пакет в окне энергий, отвечающих надбарьерному  резонансу и оцените по его ширине характерное время $\tau$. Что и почему  происходит с кривой $|\Psi (x,t)|^2$?
Почему $|\Psi|_{\max}^2$ в барьере больше, чем в пакете, который находится вне  ямы?

Сравните $\tau$ с разностью времен прохождения пакета через экран с барьером и без него. Почему в отличие от рассеяния на виртуальном уровне измеренное время задержки пакета в данном случае оказывается положительным? Какой знак имел бы аналогичный эффект в классической механике?


\hypertarget{WP-well}{}\subsection{Волновой пакет из уровней широкой
прямоугольной ямы: колебания на начальной стадии}
Для широкой ямы со многими уровнями ($n\approx50\div100$)
сформировать волновой пакет $\Psi(x,t)=\Sigma \beta_n\cdot\Psi_n
(x)\exp(-iE_nt)$ из уровней с $0>E>U/2$ на некотором интервале по
$n$, (вес гармоник $\beta_n$ автоматически задан гауссовской
функцией по $n$). Познакомиться с $|\phi_n(k)|^2$ на том же
интервале по $n$.

Рассмотреть временную эволюцию огибающей
волнового пакета  $|\Psi (x,t)|^2$ и его <<наполнения>> $\Re
\Psi(x,t)$ или $\Im \Psi(x,t)$, пока пакет не очень расплылся.
Сопроводить рассмотрение на этом начальном интервале времени
наблюдением пакета в импульсном представлении $|\phi(k,t)|^2$.

Как выглядит $|\phi(k,t)|^2$, когда пакет в координатном
представлении прижат к стенке, либо оторван от них?
Провести аналогию с классической частицей
(моменты времени, когда огибающая является гладкой).
Объяснить появление частых осцилляций в координатном
распределении, когда пакет подходит к стенкам ямы. Как (и почему
именно так) ведут себя в это время $Re \Psi(x,t)$ и $Im
\Psi(x,t)$, 3D:$\Psi(x,t)$?

\hypertarget{Revivals WP in well}{}\subsection{Расплывание и
возрождение волнового пакета в широкой прямоугольной яме}
Продолжая наблюдение на большом интервале времени (увеличив шаг
$ht$ по сравнению с предыдущим упражнением) следить за
расплыванием пакета, наступлением стадии «квантового хаоса», а
затем за появлением дробных и целых возрождений (правильной и
даже исходной формы) волнового пакета. Учитывая вид $|\phi_n(k)|
^2$ для смешиваемых волн, рассмотреть аналогию с бегунами на
стадионе, у которых скорости эквидистантны.

\hypertarget{oscillator}{}\subsection{Модель осцилляторной ямы}
Рассмотрите уровни энергии, волновые функции, импульсные
распределения и движение волнового пакета в ступенчатом
потенциале, имитирующем осцилляторную яму, обрезанную справа и
слева нулевым потенциалом. Почему наиболее высокий уровень
желательно  рассматривать отдельно?
Каково положение уровней относительно дна ямы?

Почему внешние пики $|\Psi_n(x)|^2$ больше и толще внутренних?
Почему $|\phi_n(k)|^2$ подобны $|\Psi_n(x)|^2$?
Как зависят
средняя кинетическая и средняя потенциальная энергия частицы от
$n$?

Почему волновой пакет из уровней этой ямы движется
периодически и чему равен период колебаний? (См. задачу про пакет в свободном пространстве). Как соотносятся между собой $|\Psi (x,t)|^2$ и $|\phi(k,t)|^2$?
Каким окажется период, если смешать каждое второе состояние?

\hypertarget{triangular_well}{}\subsection{Модель треугольной
(трапецевидной) ямы}
Ступеньками задайте потенциальную яму в форме
треугольника или трапеции (модель постоянного электрического
поля). Как меняется дистанция между уровнями
с ростом $n$? Почему частица на уровнях выше основного
предпочитает менее глубокую  часть ямы?

\hypertarget{Parapolic barrier}{}\subsection{Плавный барьер}
Вершина плавного барьера может считаться параболой.
Рассмотрите $T(E)$, $T(z)$ и $|\Psi (x)|^2$ в случае
перевернутого осцилляторного потенциала из предыдущего упражнения
(z-ширина барьера в основании). Чему равен коэффициент
прохождения при $E=\max U(x)$ и почему в $T(E)$ нет осцилляций, а
на графике $T(z)$ при надбарьерном прохождении они тоже сильно
подавлены? Заметим, что в Л.Л. (M.1974. стр. 220) приведено
решение для $T(E)$ в случае идеальной параболы.
%, и ступень в
%$T(E)$ имеет такую же форму, как знаменитая функция Ферми--
%распределение фермионов по энергии при разных температурах
%(только перевернутую по энергии). Какие величины в нашем случае
%формально  аналогичны уровню Ферми и температуре?

%Почему здесь
%нет надбарьерных резонансов?
Почему в <<осцилляторной>> {\em яме}
при малых E все еще есть пики $T(E)$, которые эквидистантны в
$T(z)$, т.е. наблюдаются виртуальные уровни?

\hypertarget{Landau well/barrier}{}\subsection[Модель
безотражательного потенциала]{Модель потенциала $U_0/\ch^2(x/a)$}
Рассмотрите уровни энергии, волновые функции и импульсные
распределения в ступенчатом потенциале, имитирующем яму
$U_0/ch^2(x/a)$. Почему уровни сгущаются к потолку и внутренние
пики $|\phi_n(k)|^2$ выше и уже внешних.
%Заметим, что ситуация отдаленно напоминает ридберговские уровни в атомах.

Смешайте уровни верхней части спектра (аналогично задаче про
широкую прямоугольную яму) и наблюдайте дробное и полное
возрождение волнового пакета.

Посмотрите $T(E)$, $T(a)$ для этой
ямы и барьера (замена знака $U_0$), а также $|\Psi (x)|^2$ при
небольших $E>0$ и, соответственно при $E>|U_0|$ и разных $a$.
Куда делись виртуальные уровни, которые были в случае
прямоугольной и даже параболической ямы? Заметим, что в Л.Л.
(M.1974. стр. 105) приведено решение для $T(E)$ в случае
потенциала $U_0/ch^2(x/a)$, <<хвосты>> которого у нас все же
обрезаны.


\hypertarget{double_well}{}\section{Пара ям --- модель двухатомной
молекулы}
\hypertarget{double_equal_well1}{}\subsection{Пара одинаковых
ям --- модель ковалентной связи}
Пусть расстояние между ямами соизмеримо с длиной <<хвоста>>
$Re \Psi_0(x)$. По внешнему виду $Re \Psi_n(x)$  объясните
попарную группировку  уровней и скажите из какого  уровня одной
ямы произошли те или иные уровни для  пары ям. Почему
симметричные состояния лежат немного ниже, а
антисимметричные слегка выше, чем <<родительский>> уровень?
Почему  $E_1-E_0 \ll E_3-E_2$?

По виду $|\Psi (x)|^2$ для пары нижних уровней
скажите, в каком состоянии средняя  потенциальная  энергия
частицы имеет наименьшее  (наибольшее  по  модулю) значение.
Почему для полных энергий $E_n$ получается наоборот?

По давлению частицы на  внешние и внутренние потенциальные стенки
скажите  притягиваются   ли  ямы  друг  к   другу?  В  каких
состояниях  частица  стремится  сблизить  ямы?  (В химии они
называются связывающими молекулярными орбиталями).
В каких состояниях, наоборот, частица расталкивает ямы (они
называются антисвязывающими).

\hypertarget{double_equal_well1_k}{}\subsection{Импульсное
распределение в двух ямах}
Объясните вид $|\phi_n(k)|^2$ по зависимостям $Re
\Psi_n(x)$  (наличие  двух  побочных  максимумов  для  основного
состояния,  близость характерных импульсов для $n=1$,$2$,
совпадение положений  побочных максимумов  для $n=0$ с положением
основных максимумов для $n=3$).

Объясните  вид графиков $|\phi_n(k)|^2$ для  двух нижних уровней
по  аналогии с дифракцией  частиц на открытом  конце
пары  связанных   плоских  волноводов  или   на  двух  щелях
($k_x/k_y=\tg(\theta)$,  где $y$--вдоль волновода, $\theta$ --
угол  дифракции). Рассмотрите разность хода от центров щелей до
точки на экране и по виду $Re \Psi(x)$ покажите, что будет
наблюдаться необходимое чередование черных и белых полос.

Какой была  бы картина дифракции $|\phi(k)|^2$,  если бы ямы
находились далеко  друг от друга?  Как соотносятся огибающая
кривой  почернения с $|\phi(k)|^2$ для  основного состояния в
одной яме?

\hypertarget{double_equal_well_z}
{}\subsection{Однопараметрическое изменение симметричной пары ям}
Что произойдет с уровнями, волновыми функциями и давлением
частицы на стенки, если увеличить дистанцию между ямами в 2-3
раза? Предскажите и проверьте с помощью $E_n(z)$  изменение
спектра уровней с при  раздвигании ям. Укажите на этих графиках
силу, с которой частица притягивает (отталкивает)
ямы. Найдите на рисунке связывающие и антисвязывающие
<<молекулярные орбитали>>.

Предскажите изменение спектра уровней при отрастании узкого
барьера  между  одинаковыми  ямами.  Почему антисимметричные
состояния неподвижны?

Предложите двух-ямный  потенциал, с локализацией  частиц на
разделяющем  промежутке  (в  каком  либо  из  уровней).  Как
соотносится этот эффект с резонансами при прохождении частиц
через  прямоугольный   барьер? Меняя  ширины  ям  и  разделяющего
промежутка, рассмотрите процессы  столкновения уровней  с
локализацией в ямах  и в барьере.

\hypertarget{WP_double_well}{}\subsection{Осцилляции во времени}
Для двух одинаковых ям рассмотрите временную эволюцию
суперпозиции пары уровней $\Psi_m(x)\exp(-iE_mt)+
\Psi_n(x)\exp(-iE_nt)$. Почему  движение периодично и сильно
различаются характерные времена колебаний $Re \Psi(x,t)$  и $|
\Psi (x,t)|^2$? Будет ли частица излучать  электромагнитные
волны, если смешать 0-ой и 1-ый (0-ой и 2-ой) уровни?

\hypertarget{double_inequal_well}{}\subsection{Пара разных ям ---
модель ионной связи в двухатомной молекуле}
Объясните эффект локализации волновой функции %частицы
и смены области локализации с ростом $n$ в слегка асимметричной
паре ям. Почему эффект  сильнее для нижних уровней?
К чему ведет подобная локализация в реальности, если верхний из
заполняемых уровней в разных атомах, рассмотренных по
отдельности, недозаполнен и в одном из атомов лежит выше, чем в
другом (разные потенциалы ионизации)?
Объясните зависимость $E_n$ от расстояния между ямами.

Рассмотрите  изменение  уровней при неизменной одной яме и
постепенном уменьшении глубины соседней более глубокой ямы
до нуля. Наблюдайте столкновение и <<антипересечение>> уровней,
отвечающее сначала равенству ям, а потом их сильному различию.
Как выглядят волновые функции в момент такого
антипересечения? Почему, и при каких условиях на положение
уровней отдельных атомов, это позволяет ковалентную связь.

При какой  минимальной разности глубин  сильная локализация
частиц в ямах исчезнет? Рассмотрите  изменение  уровней
энергии  при встречном изменении глубин исходно разных ям
(<<поляризация>> внешним полем, меняющимся от $\epsilon$ до
$-\epsilon$ --- линейный эффект Штарка при большом поле и
параболический вблизи $\epsilon=0$).

\hypertarget{N_well}{}\section[Несколько ям --- модель
кристалла]{Несколько ям --- модель <<кристалла>>}
\hypertarget{N_equal_well}{}\subsection{Несколько одинаковых ям}

Пусть ширина барьера между одинаковыми эквидистантными ямами
соизмерима с длиной <<хвоста>> $Re \Psi_0(x)$. Объясните наличие
<<зон>> разной ширины в  спектре уровней.

Оценивая по виду $Re \Psi(x)$  волновые числа $k$, покажите, что
энергия  основного  состояния  понижается,  если вместо двух
одинаковых  ям   взять  несколько  таких   же  эквидистантно
расположенных ям. Покажите что верхний край нижней зоны лежит
выше, чем  второй  уровень   для  двух  ям. Покажите, что
при нечетном числе ям средний по номеру уровень в <<зоне>> почти
совпадает с уровнем в одной яме.

Исходя из сил в точках разрыва потенциала
скажите для каких краев зон частица притягивает (расталкивает)
ямы. Почему сходны $|\Psi_n(x)|^2$ для  состояний
равноотстоящих  по  $n$  от  краев  <<зоны>> (например:  второго  и
предпоследнего в зоне)?

Исходя из формы $Re \Psi_n(x)$ объясните
положение пиков импульсного распределения для краев зон и для
состояний внутри нижней зоны. Как связана ширина пиков с числом
ям?

\hypertarget{continuous to band}{}\subsection{Переход от
непрерывного к зонному спектру}
Нарисуйте зависимость  спектра уровней от  расстояния между
ямами. В каких состояниях частица стремится сблизить
(оттолкнуть) ямы? Что при этом происходит со средней
потенциальной (кинетической) энергией электронов ?

Почему Na легко образует кристалл, а Ne нет?

Рассмотрев изменение спектра  при отрастании гребенки узких
барьеров  на   дне  широкой  ямы,   объясните  неподвижность
некоторых состояний и образование зон. Образуются ли зоны,
если барьеры заменить ямами ?

\hypertarget{Tamm_state}{}\subsection{Таммовский уровень}
Проследите на волновых функциях и $E_n$ что происходит с
состояниями зон в $N$ ямах, когда увеличивается $U(x<0)$? (Модель
поверхности кристалла с большой работой выхода электронов). А
если сделать наоборот? Объясните появление локализации.
Почему
делокализованные состояния <<избегают>> яму возле $x=0$?

Сравните
волновые функции  первого и второго <<поверхностных состояний>>,
почему затухание вглубь кристалла будет разным?

Как (и почему именно так) выглядят зависимости уровней от
расстояния между ямами? Образуются  ли  поверхностные   состояния
в случае узких барьеров?  Рассмотрите  столкновение  таммовского
уровня с <<объемными>> состояниями при сближении ям. Найдите
поверхностные состояния для гребенки узких ям на дне широкой
ямы.


\hypertarget{Impur_state}{}\subsection{Модель примесной ямы в
длинной молекуле}
Слегка  расширьте (углубите)  или  сделайте  уже (мельче)
одну  из  нескольких одинаковых ям,  объясните изменения спектра
и волновых функций, которые при этом произойдут. Рассмотрите
случаи с симметричным и несимметричным  расположением  <<примесной
ямы>>. Объясните эффекты  попарной группировки  уровней  в  зоне и
смены области локализации частицы с повышением $n$.
Почему делокализованные состояния <<избегают>> примесную яму.

\hypertarget{Stark_state}{}\subsection{Штарковская лестница}
Рассмотрите спектр, $Re\Psi_n(x)$, $|\Psi_n(x)|^2$, $|\phi_n(k)|
^2$, когда на <<конечный кристалл>> действует электрическое поле.
Объясните эквидистантность спектра уровней, произошедших,
например, из нижней зоны, и почему межуровневая дистанция
больше для крайних номеров, чем для внутренних? Проследите
за преобразованием зонного спектра в штарковские лестницы на
графиках $E_n(z)$, где $z$ --- электрическое поле.

\hypertarget{WP_Stark_state}{}\subsection[Блоховские осцилляции]
{Блоховские осцилляции}
Сформируйте волновой пакет из всех состояний штарковской
лестницы, возникших из нижней зоны кристалла из $N$ ям.
Наблюдайте колебания $|\Psi_(x,t)|^2$, $|\phi_n(k,t)|^2$, и
объясните их периодичность. Почему с одной скоростью перемещаются
пики импульсного распределения.

\hypertarget{quasilevel}{}\section{Квазиуровни}
\hypertarget{quasilevel_two_barrier}{}\subsection{Пара барьеров}
Найдите классически запрещенные состояния полной прозрачности
для пары одинаковых нешироких барьеров (по сравнению
с характерной длиной затухания волновой функции).
Какой вид имеет волновая функция для этих резонансов внутри
и между барьерами? Почему вероятность найти частицу  в квантовом
колодце резко возрастает при резонансной энергии? Что происходит
с временем жизни частицы в резонаторе при увеличении толщины
барьеров? Как  связаны  энергии  резонансов  с  положением
связанных состояний в яме, ширина которой равна расстоянию между
барьерами, а глубина совпадает с их высотой?

Что происходит с $T(E)$ и резонансными $|\Psi(x)|^2$, когда один
барьер меняет свою высоту (толщину)? Что будет, если различные
барьеры поменять местами?

\hypertarget{three_barrier}{}\subsection{Три барьера}
Как изменится график $T(E)$, если  перейти от двух барьеров к
трем? Какой вид будут иметь $|\Psi(x)|^2$ для нижнего и
следующего
квазиуровней, а также для  точки  минимального  $T$  между  ними?
Почему в последнем случае частица локализуется в одной из ям?
Почему залечивается данный провал в $T(E)$, если крайние барьеры
сделать в два раза уже внутреннего (тройник Иогансена)?

\hypertarget{N_barrier}{}\subsection{Несколько барьеров}
Исследуйте  образование <<зонной>>  структуры квазиуровней  с
увеличением числа барьеров. Объясните эффекты сгущения и сужения
резонансов к краям нижней <<зоны>> квазиуровней.

\hypertarget{band_N_well_positive_E}{}\section[Зоны для
нескольких ям при положительных энергиях]{<<Зоны>> для нескольких
ям при $E>0$}
Найдите свидетельства образования <<зон>> в непрерывном спектре
нескольких одинаковых ям. Как меняется глубина провалов $T(E)$
между соседними <<зонами>> резонансов  при изменении числа
одинаковых ям? Как выглядит волновая  функция в области этих
провалов (главных максимумов брэгговского отражения при
нормальном падении волн)?

Сравните между собой кривые $T(E)$ внутри разрешенных зон при
разных числах ям (или барьеров). Что  можно сказать  о нижних
огибающих ?  Какова связь минимумов $T(E)$ в разрешенной <<зоне>> с
резонансами на разных частях потенциальной гребенки?

\hypertarget{quasistacionary_state}{}\section{Квазистационарные
состояния}
\hypertarget{quasistacionary_state_two_barrier}{}\subsection{Пара
барьеров}
Найдите дискретные комплексные энергии $E_n+iG_n$ распадных
(квазистационарных) состояний для двух барьеров. В чем  отличие
этих состояний от квазиуровней в стационарном
рассеянии, если судить по виду $|\Psi(x,t=0)|^2$ и $Re
\Psi(x,t=0)$? (Сравните направление движения волн). Разберитесь
по описанию в  Инструкции к программе в гранусловиях,
отвечающих состояниям для произвольных $E+iG$, $G<0$, а также
в двух процедурах поиска дискретных квазистационарных
состояний. Сравните также положение и ширину распределений
Брейта--Вигнера с $T(E)$ для этих же барьеров.

Рассмотрите временную эволюцию на графиках $|\Psi(x,t)|^2$, $Re
\Psi(x,t)$, 3D:$\Psi(x,t)$ отдельно для нижнего и следующего
квазистационарных состояний. В чем проявляется уход волн от
области размещения барьеров? (По динамике плотности вероятности
сравните эти случаи со случаем суперпозиции двух
квазистационарных состояний).

Что происходит с квазистационарными состояниями при изменении
ширины (глубины) одного из барьеров?

\hypertarget{quasistacionary_state_N_barrier}
{}\subsection{Несколько барьеров}
Сделайте предыдущее упражнение для нескольких барьеров.
Как размещены на плоскости (E,G) квазистационарные состояния,
отвечающие уровни нижней и следующим зонам квазиуровней?
На каких величинах и графиках, и как именно, проявляется более
трудный распад крайних состояний в зоне?
%Почему залечивается данный провал в $T(E)$, если крайние барьеры
%сделать в два раза уже внутреннего (тройник Иогансена)?

\hypertarget{quasistacionary_state_well}
{}\subsection{Квазистационарные
состояния для ямы/барьера}
Исследуйте квазистационарные состояния для одного прямоугольного
барьера. Покажите, что они аналогичны надбарьерным резонансам
стационарного рассеяния.

Рассмотрите квазистационарные состояния для одной прямоугольной
ямы. Как связаны они с виртуальными уровнями в стационарном
рассеянии? Найдите случаи, когда действительная часть энергии
квазистационарного состояния становится отрицательной, но мнимая
отлична от 0. Сравните в этом случае распределение Брейта--Вигнера
с $T(E)$ для той же ямы и объясните возможность таких распадных
состояний. Как они соотносятся со связанными состояниями.

\hypertarget{quasistacionary_state_LL}
{}\subsection[Квазистационарные состояния для ``безотражательной''
ямы]{Квазистационарные состояния для ямы $U_0/\ch^2\left(\frac
xa\right)$}
Ямы вида $-U_0/\ch^2(x/a)$ обладают следующим замечательным свойством:
при $U_0a^2=N(N+1)$ они совершенно прозрачны для частиц любой энергии
$E>0$, а при других $U_0a^2$ отражение  экспоненциально быстро и без
осцилляций по $E$ стремится к нулю (Л.Л., M.1974. стр. 105). Таким образом,
кажется у этой
ямы нет виртуальных уровней. Предлагается рассмотреть
квазистационарные состояния и прояснить эту ситуацию
построением распределений Брейта--Вигнера на фоне $T(E)$.
Проверьте, что  значения $E_n+iG_n$ слабо зависят от
числа ступеней, имитирующих  данный потенциал, если оно велико.
Есть ли квазистационарные состояния для барьера той же формы?
Если да, то как это согласуется с отсутствием резонансов в T(E)?

\hypertarget{Periodical_potentials}{}\section{Периодический
потенциал}

\hypertarget{Periodical_1well}{}\subsection{Одна яма на периоде}
Задайтесь характерным периодическим потенциалом: пусть период
содержит одну прямоугольную яму с малым числом уровней и рядом с
ней имеется отрезок нулевого потенциала, ширина  которого
соизмерима с $1/\sqrt E_0$. Посмотрите расположение краев зон
$E_n$ на интервале от дна  ямы до достаточно большого
$E_{\max}>0$. Как меняются  ширины разрешенных и запрещенных  зон
с ростом $E$?

Полезно сравнить положение зон с результатами решения задач про
конечное число ям. По аналогии с этими задачами исследуйте, в том
числе на зависимостях $E_n(z)$, формирование и трансформацию
зонного спектра при изменении глубины (ширины) ямы при
фиксированной ширине разделяющего промежутка с $U=0$, изменении
ширины промежутка $b$ между неизменными ямами, а также изменении
$U$ на тонких ($b\ll1$) разделяющих промежутках от глубоких
отрицательных до больших положительных значений при
условии $|U|b^2\ll1$.

Вернитесь к характерному периодическому потенциалу. Какой  вид
имеют $Re \Psi(x)$, $Im \Psi(x)$, 3D:$\Psi(x)$ для краев зон,
внутри разрешенных и запрещенных зон? Где на этих графиках
квазиимпульс? Чему он равен для краев зон? Какие значения имеет
квазиимпульс в запрещенной зоне? По форме $Re \Psi(x)$
предскажите   вид   импульсного распределения для краев зон. Чем
(и почему) оно отличается внутри  разрешенных зон от $|\phi_n(k)|
^2$ в случае конечного числа одинаковых ям? Где находится
квазиимпульс на графике импульсного распределения в случае
периодического поля.

Изобразите $qa(E)$ ---  зависимость квазиимпульса от энергии
(закон дисперсии). Как определяются по закону дисперсии средняя
скорость и эффективная масса частицы?
Как они меняются с ростом номера разрешенной зоны и внутри зон?
Сформируйте волновой пакет из состояний в разрешенной зоне и
сопоставьте направление движения максимума $|\Psi(x)|^2$ с
номером зоны, направлением квазиимпульса, положением основного
пика импульса и направлением средней скорости.

\hypertarget{Periodical_Nwell}{}\subsection{Несколько ям на
периоде}
Задайте вместо одной две (потом три) одинаковые прямоугольные ямы
на периоде и объясните форму закона дисперсии, посмотрите, как
изменится импульсное распределение для краев зоны и его поведение
с изменением $E$ внутри зон. Рассмотрите и объясните форму и
движение волнового пакета, сформированного в таком же окне
энергий, как для одной ямы на периоде. Что произойдет с
разрешенными зонами, если слегка изменить потенциал на каком-либо
отрезке или ширину этого отрезка? Меняя это отклонение в
противоположные стороны задайте соответственно начальный и
конечный потенциал, постройте и объясните зависимость $E_n(z)$.

\end{document}
